% vim: spell spelllang=en_gb
\chapter{Conclusion}%
This project shows that Twitter can be a great data source candidate to facilitate disaster
management tasks. The pipeline extracts information about flood events using the following steps:
(1) extracting flood-relevant tweets, (2) identifying geographical locations, (3) finding insights
through text analysis, and (4) presenting the results using a visual interface. Even though the
methods have room for improvement, they can extract relevant information about past flood events,
showcasing the potential of knowledge extraction from social media for disastrous events.

Natural disasters impact human lives severely and will not disappear; they will only worsen due to
climate change. With that said, people can try to reduce its impact by preparing for it and
repairing the damages it made after dissipating, which is possible by using social media as a data
source to predict and analyse events. One problem with this approach is the lack of people's
participation, making the amount of data limited, thus leading to inaccurate results or the absence
of information to reach them. If this framework gets established globally and people become aware of
it, they will be more inclined to share their knowledge on social media to enhance its results. It's
a solution for the people and by the people.

