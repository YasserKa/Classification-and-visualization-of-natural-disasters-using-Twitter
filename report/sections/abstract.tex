% vim; spell spelllang=en_gb
\section*{Abstract}

The more established information about a disaster, the more efficiently the disaster management is
done by the concerned parties to handle the situation. People tend to share their experiences during
disastrous events using social media, making them potential data sources. This thesis project
implements a pipeline to extract knowledge from Twitter about flood events. It determines
flood-relevant tweets using a classifier and identifies geographical locations mentioned in the
tweets using a hybrid geoparsing approach. At the end of the pipeline, the spatial, temporal, and
textual aspects of the results are presented using an interactive visual interface. The implemented
pipeline is exemplified using historical tweets created during past flood events.

