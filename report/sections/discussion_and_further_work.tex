% vim: spell spelllang=en_gb
\chapter{Discussion and Further Work}\label{sec:discussion_and_further_work}
This section discusses the results of the methods used and their reliability in addressing the
research questions mentioned in the introduction section while suggesting improvements to enhance
the results.

The more tweets obtained for an event, the more information that can be extracted about it, and the
results of the experiments show that the flood event in Gävle has more tweets than the other ones;
this could be because the event is more impactful on the society causing it to attract the attention
of the affected citizens. More data about the events can be obtained by adding more data sources
, such as \ac{GDELT}\footnote{https://www.gdeltproject.org/}, and other social media. The
pre-processing used fail to filter out near-identical tweets which are most likely an indication
that by bots for malicious or utility (e.g. acting as a feed generator) reasons. Reducing the amount
of spam can be done by blacklisting user accounts used as spam bots which can be identified by
checking for suspicious activities in their past tweets.

The evaluation metrics for the classifier based on the DistilBERT transformer seems promising, with
an accuracy of 92.31\% and F$_{1}$ score of 91.81\%; the experiments show that the model has high
precision yet a low recall which could be a side effect of the tweets being microblogs (i.e. small
text documents), where there is not enough context in the tweets to be able to classify them correctly
(e.g. ``It's wet in location X'', and ``It rained a lot yesterday''); also, since the context of the
tweets is constructed using other elements than text, such as images, hashtags, and \ac{URL}s, the
classifier will not be able to indicate the actual intent of the user, failing in categorizing the
tweet correctly (e.g. It's very wet here https://t.co/PcroA3s1A2); putting
these elements into consideration while classifying the tweets is possible by using an image
classifier and a web scraper to handle the images and the \ac{URL}s, respectively.

The pipeline identifies the locations of the events used in the experiments correctly, which is
evident from the number of tweets mentioning the location shown in the map; yet, it is unable to
identify the correct geographical locations for some keywords, such as Spain and Turkey. Some terms
can refer to multiple locations existing in the world, and the most likely referred location can be
identified by generating a confidence score using several factors, such as the
``importance''\footnote{https://nominatim.org/release-docs/develop/customize/Importance/} attribute
obtained from the Nominatim package. Another improvement to the location extraction step would be
using a better way to handle the existence of several locations in one tweet instead of picking the
one with the smallest parameter only.

The plots in the visual interface present the textual, spatial, and temporal aspects of the
tweets in a searchable and straightforward way. The map shows the distribution of the geographical
locations mentioned in the tweet and enables filtering using regions; yet, the map does not allow box
or lasso selections, and the pop-up of the pointers shows the name of the location only, where it
could be more informative by including more information related to the tweet. The histogram shows
the temporal distribution for the creation date of the tweets, where the number of tweets is the
highest at the start of the event then it reduces gradually afterwards. This information can be a
factor in calculating the impact of the flood on society since the event attracts more attention the
more it influences the citizens. The tweets table provides a way to check some features of the
selected tweets, but it is not enough to make any direct insights even with the help of \ac{t-SNE}'s
scatter plot, \ac{LDA} table, and \ac{TF-IDF} table. Some of the clusters in the scatter plot are
made of near-identical tweets, where most of them are created by spambots. Using other
clustering techniques in the scatter plot, such as K-means, might bring better results. The weights
and frequency of terms in \ac{LDA} and \ac{TF-IDF} tables do not show any interesting patterns
either. Obtaining more data and changing the pre-processing approach might improve the results of
these techniques. More text analysis techniques can be used, such as sentiment analysis, which gives
the ability to quantify the impact of the event.


Further work can include the following:
\begin{itemize}
  \item Applying the pipeline to other type of events, such as earthquakes, by changing the query
    and the training dataset for the classifier.
\item Applying the pipeline to other countries by changing the map used in the visualization.
\item Using streaming for live event detection to identify flood events by using some criterion,
  such as sudden bursts of tweets talking about flooding.
\item Augmenting warning systems pipeline by including this project's pipeline to detect and visualize flood events.
\end{itemize}
