% vim: spell spelllang=en_gb
\chapter{Discussion and Further Work}\label{sec:discussion_and_further_work}

This section discusses the results of the methods used and their reliability in addressing the
research questions mentioned in the introduction section while suggesting improvements to enhance
the results.

The more tweets obtained for an event, the more information can be extracted about it, and the
results of the experiments show that some flood events have more tweets than others; one possible
reason is that these events are more impactful on society, causing them to attract the attention of
the affected citizens. More data about the events can be obtained by adding more data sources, such
as \ac{GDELT}\footnote{https://www.gdeltproject.org/}, and other social media. The pre-processing
used filters out tweets with the same text verbatim, leaving out near-identical tweets generated by
bots for malicious or utility (e.g. acting as a feed generator). Some tweets are flood-relevant yet
fake to provide the public with misleading information for malicious reasons and will pollute the
data. One potential improvement to the pipeline is to include methods to detect and filter out fake
news to preserve the integrity of the data. A simple approach to limit the amount of spam and fake
news is to blacklist user accounts with suspicious by checking for suspicious activities in their
past tweets and network. Another side effect of using social media is that some events happen
without getting reported because they are either out of sight from the public eye, or didn't attract
sufficient attention. Social media won't provide sufficient data, if any, about these events to
process; integrating more data sources into the pipeline, such as meteorological data, articles, and
reports from governmental agencies might supplement this deficiency.

The evaluation metrics for the classifier based on the DistilBERT transformer seem promising on the
datasets it was trained on, with an accuracy of 92.31\% and F$_{1}$ score of 91.81\%; yet, the
experiments show that the model has high precision and low recall. The nature of the text in tweets
makes it more difficult for them to get classified. The tweets are
microblogs (i.e. small text documents), where there is not enough context in the tweets to be able
to classify them correctly (e.g. ``It's wet in location X'', and ``It rained a lot yesterday'');
also, since the context of the tweets is constructed using other elements than text, such as emojis,
images, hashtags, and \ac{URL}s, the classifier will not be able to indicate the actual intent of
the user, failing in categorizing the tweet correctly (e.g. It's very wet here
https://t.co/PcroA3s1A2); putting these elements into consideration while classifying the tweets is
possible by using an image classifier and a web scraper to handle the images and the \ac{URL}s,
respectively. There's a drawback to using a classifier that works only on English text, since it
makes the translation step necessary, influencing the quality of the data, and ultimately impacting
the performance of the trained model. Using a classifier that accepts Swedish text will not require
the text to be translated.

The pipeline identifies the locations of the events used in the experiments correctly, which is
evident from the number of tweets mentioning the location shown in the map; yet, it is unable to
identify the correct geographical locations for some keywords, such as Spain and Turkey. Some terms
can refer to multiple locations existing in the world, and the most likely referred location can be
identified by generating a confidence score using several factors, such as the
``importance''\footnote{https://nominatim.org/release-docs/develop/customize/Importance/} attribute
obtained from the Nominatim package. Another improvement to the location extraction step would be
using a better way to handle the existence of several locations in one tweet instead of picking the
one with the smallest parameter only.

The plots in the visual interface present the textual, spatial, and temporal aspects of the tweets
interactively. The map shows the distribution of the geographical locations mentioned in the tweet
and enables filtering using regions; yet, the map does not allow box or lasso selections, and the
pop-up of the pointers shows the name of the location only, where it could be more informative by
including more information related to the tweet. The histogram shows the temporal distribution for
the creation date of the tweets, where the number of tweets is the highest at the start of the event
then it reduces gradually afterwards. This information can be a factor in calculating the impact of
the flood on society since the event attracts more attention the more it influences the citizens.
The tweets table provides a way to check some features of the selected tweets, and it could be
improved by adding a text filter to focus on tweets that contain specific terms; this feature would
support the ``Filter'' task for the textual data. The test cases show a potential use case for
textual filtering by limiting the tweets to the terms related to trends found by exploring the
clusters in the \ac{t-SNE} scatter plot, such as traffic disruption and \ac{SMHI} warnings. The
clustered 2-dimensional space provides a method to find similar tweets using their spatial
proximity, potentially referring to similar events, assessed using the spatio-temporal
distribution in the map and histogram plots. Besides these trends, the results of the text analysis
techniques shown in the \ac{t-SNE}'s scatter plot, \ac{LDA} table, and \ac{TF-IDF} table didn't
provide any useful insights due to the nature of the text. Using other clustering techniques in the
scatter plot, such as K-means, might bring better results. Obtaining more data and changing the
pre-processing approach might improve the results of these techniques. More text analysis techniques
can be used, such as sentiment analysis, which gives the ability to quantify the impact of the
event.

Further work can include the following:
\begin{itemize}
  \item Applying the pipeline to other type of events, such as earthquakes, by changing the query
    and the training dataset for the classifier.
  \item Applying the pipeline to other countries by changing the map used in the visualization.
  \item Using streaming for live event detection to identify flood events by using some criterion,
    such as sudden bursts of tweets talking about flooding.
  \item Augmenting warning systems pipeline by including this project's pipeline to detect and visualize flood events.
\end{itemize}
