\section{Introduction}

In early twenties, floods around Lake Vänern and Arvika have costed Sweden an estimate of 11.1
billions Swedish Krona for damages and repairs \cite{RiverFloodsSweden2022}. Counties of Dalarna and
Gävleborg has suffered from flash floods in 2021 disturbing the daily life of their citizens and
damaging public and private properties \cite{daviesSwedenFlashFloods2021}.  Flooding is a
devastating natural disaster that threatens the lively hood of people and the infrastructure of
communities around the world \cite{Floodlist2021}.

To facilitate the process of emergency management during these hazardous events, early warning
systems analyze their risk, monitor and warn the public while ensuring their readiness
\cite{contributorsEarlyWarningSystem2022}. Traditionally, meteorologists forecast the weather by
relying on tools such as gauges, satellites, and radars for data extraction. The emergence of social
media platforms such as twitter provide individuals a public space to share their experience,
effectively creating another potential source of data.

Researchers started harnessing this new wealth of information to aid the disaster management
procedure.  Twitter's stream API makes it possible to create a monitoring system for early event
detection on a global \cite{debruijnGlobalDatabaseHistoric2019b} and local
\cite{barkerDevelopmentNationalscaleRealtime2019} scales. Another use for it would be identifying
victims in real time, locate their physical location, and communicate the information to rescue
teams \cite{singhEventClassificationLocation2019c}. After the threat subsides, emergency
managers can use relevant tweets to assess the impact and plan the recovery phase
\cite{barkerDevelopmentNationalscaleRealtime2019}. To prepare for future floods, authoritative
entities can make informed actions by analyzing historical data and determine the locations
suffering from recurrent calamity. This new acquired knowledge is able to augment weather warning systems'
pipelines improving their accuracy \cite{nesetAI4ClimateAdaptation}.

This thesis project implements a pipeline that provides a visual representation of tweets related to
flood events in Sweden. First, relevant tweets are pulled, processed, and classified from the twitter
API using data mining techniques. Second, physical locations are extracted from tweets mentioning flood
events employing \ac{NER} and gazetteer. Finally, the identified locations with
relevant information from tweets are presented on a spatio-temporal visualization. For verification
purposes, the pipeline is applied on a week worth of tweets after past flood events.


 % \begin{figure}[ht]
 %    \centering
 %    \caption{Dummy figure} 
% \end{figure}

